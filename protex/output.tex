%                **** IMPORTANT NOTICE *****
% This LaTeX file has been automatically produced by ProTeX v. 1.1
% Any changes made to this file will likely be lost next time
% this file is regenerated from its source. Send questions 
% to Arlindo da Silva, dasilva@gsfc.nasa.gov
 
%------------------------ PREAMBLE --------------------------
\documentclass[11pt]{article}
\usepackage{amsmath}
\usepackage{epsfig}
\usepackage{hangcaption}
\textheight     9in
\topmargin      0pt
\headsep        1cm
\headheight     0pt
\textwidth      6in
\oddsidemargin  0in
\evensidemargin 0in
\marginparpush  0pt
\pagestyle{myheadings}
\markboth{}{}
%-------------------------------------------------------------
\setlength{\parskip}{0pt}
\setlength{\parindent}{0pt}
\setlength{\baselineskip}{11pt}
 
%--------------------- SHORT-HAND MACROS ----------------------
\def\bv{\begin{verbatim}}
\def\ev{\end{verbatim}}
\def\be{\begin{equation}}
\def\ee{\end{equation}}
\def\bea{\begin{eqnarray}}
\def\eea{\end{eqnarray}}
\def\bi{\begin{itemize}}
\def\ei{\end{itemize}}
\def\bn{\begin{enumerate}}
\def\en{\end{enumerate}}
\def\bd{\begin{description}}
\def\ed{\end{description}}
\def\({\left (}
\def\){\right )}
\def\[{\left [}
\def\]{\right ]}
\def\<{\left  \langle}
\def\>{\right \rangle}
\def\cI{{\cal I}}
\def\diag{\mathop{\rm diag}}
\def\tr{\mathop{\rm tr}}
%-------------------------------------------------------------

\markboth{Left}{Source File: sao\_rt\_tool\_introduction.txt,  Date: Fri Apr  5 16:50:55 EDT 2013
}

 
 
%/////////////////////////////////////////////////////////////
\newpage

\markboth{Left}{Source File: GC\_parameters\_module.f90,  Date: Fri Apr  5 16:50:55 EDT 2013
}

 
%/////////////////////////////////////////////////////////////
\title{Radiative Transfer Tool}
\author{{\sc Gonzalo Gonzalez Abad and Xiong Liu}\\ {\em Smithsonian Astrophysical Observatory, Atomic and Molecular Physics Division}}
\date{April, 2013}
\begin{document}
\maketitle
\tableofcontents
\newpage
\section{Routine/Function Prologues} \label{app:ProLogues}

  \subsection{Fortran:  Module Interface GC\_parameters\_module.f90 }


  This module contains dimensioning parameters, 
   angle conversion and PI definition.
  \\
  \\{\bf INTERFACE:}
\begin{verbatim} MODULE GC_parameters_module.f90
   IMPLICIT NONE\end{verbatim}{\bf PUBLIC DATA MEMBERS:}
\begin{verbatim}    Dimensioning
    ------------
 
    Dimensions with 'GC_' prefix, distinguish from VLIDORT variables
 
    integer, parameter :: GC_maxlayers      = 51
    integer, parameter :: GC_maxgeometries  =  1
    integer, parameter :: GC_maxuserlevels  =  1
 
    wavelengths, gases, messages
    integer, parameter :: maxgases      = 10
    integer, parameter :: maxaer        = 6
    integer, parameter :: maxcld        = 3
    integer, parameter :: maxflambdas   = 53501
    integer, parameter :: maxlambdas    = 2700
    integer, parameter :: maxmessages   = 100
    integer, parameter :: maxmoms       = 32
    integer, parameter :: maxgksec      = 6, maxgkmatc = 8
    integer, parameter :: maxscatter    = 3  ! Molecules, aerosols, clouds
    INTEGER, DIMENSION(maxgkmatc), PARAMETER :: &
         greekmat_idxs = (/1, 2, 5, 6, 11, 12, 15, 16/), &
         phasmoms_idxs = (/1, 5, 5, 2, 3, 6, 6, 4/)
    real(KIND=8), PARAMETER :: pi      = 3.14159265358979d0
    real(KIND=8), PARAMETER :: deg2rad = pi / 180.d0,    &
                               rad2deg = 180.d0 / pi
 \end{verbatim}{\bf REVISION HISTORY:}
\begin{verbatim}    April 2013 - G. Gonzalez Abad - Initial Version\end{verbatim}

%...............................................................
\end{document}
